\chapter{Fundamentals}\label{chap:fundamentals}

In this chapter, I first introduce the basics of localization. Afterwards, I give an overview of different localization algorithms being used in related work. At the end of this chapter, I focus on the \acl{PF} and the \acl{MCL} algorithm, which I have chosen for my implementation. The detailed reasons for choosing \acs{MCL} for this project are pointed out in section~\ref{sec:algo_decision}.

\section{Fundamental Aspects of Localization}\label{sec:fund_loc}

Localization, i.e.\ positioning, is the process of estimating the position of an object in its environment. Usually the object's environment is being defined by using a coordinate system. One of the most popular coordinate systems is the \emph{World Geodetic System 1984 (WGS84)}, which describes a position on earth, by longitude and latitude. Sometimes, it is sufficient to use a local coordinate system; for instance, a two or three dimensional cartesian coordinate system. Indoor localization is one example, where it is sufficient to use a local coordinate system, because the object's position is limited to a defined area, e.g.\ within a building.

\subsection{Location Types}
There are four types of locations. A \emph{physical location} is described by coordinates on a map. \emph{Symbolic locations} such as ``in the living room'' or ``on the small table in the kitchen'', are often used by humans. If all objects are sharing a frame of reference, like chess pieces on a chess board, the type of location is called an \emph{absolute location}. When an object has its own reference frame, as for instance when saying ``I am two meters away from the door'', which often includes a proximity value, it is called a \emph{relative location} \citep{IEEE:survey_wireless_indoor_pos}.

\subsection{System Topology}
Usually, a pure wireless localization system consists of at least one transmitter that emits some sort of signal, and one or more receivers, i.e.\ measuring units. The system's topology can be \emph{remote positioning}, which means that the transmitter's position is being estimated by using the measurements from multiple measuring units, that are deployed at fixed locations. The position estimation takes place in a component separate from the transmitter. This component uses the measurements as input. If the measurement unit is mobile and capable to estimate its own position by collecting the signals from fixed transmitters, the system's topology is called \emph{self-positioning}. \emph{Indirect remote positioning} is basically the same as self-positioning, but the collected measurements are sent via some data connection to an external service, which performs the position estimation. If the system's topology is basically remote positioning, but the measurements are transferred via a wireless data link to the mobile side, the topology is called \emph{indirect self-positioning} \citep{IEEE:survey_wireless_indoor_pos}.

Besides a pure wireless localization systems, which just uses wireless signals for the positioning, other systems, often used in the area of mobile robotics, do exist. Robots are often equipped with sensors, such as laser scanners or ultrasonic sensors, to measure distances to obstacles. Wheel- and chain-based robots are usually also equipped with odometers, others have accelerometers and rotation rate sensors to estimate the traveled distance. The used localization algorithms are able to combine different sensors to provide a more accurate estimation \citep{thrun:prob_robo}. Mobile robots, especially if they operate autonomously, are usually self-positioning, or indirect remote positioning systems, which often depends on their system's hardware capabilities. Remote positioning as well as indirect self-positioning systems are rarely the case in the field of mobile robots.

\subsection{Uncertainty}
In order to derive an accurate position, localization systems need to be able to accommodate uncertainty. According to \citet{thrun:prob_robo}, uncertainty in robotic systems has several reasons.

One of these reasons is the \emph{environment}. If a robot works together with humans, the environment is very dynamic and unpredictable, because the robot does not know for; instance, where and when a human moves to.
Of course, robots are equipped with \emph{sensors} to recognize obstacles like humans, but sensors may also cause uncertainty. On one hand, they have limitations, e.g.\ a camera has a specific resolution, and on the other hand, noise may unpredictably influences their measurements. 
Typically, robots have \emph{actuators}; for instance, motors to move them within their environment. Actuators have precision limitations. Additionally, their successive physical components; for instance, a robot's wheels on different surfaces, may cause unpredictable effects.
To allow robots to operate in a physical environment, models of it are being used. \emph{Models} are abstractions, which approximate a certain object or behavior. Thus, information is being lost, and additionally every model contains some inaccuracy. Robotic systems work in real-time environments. Due to computational reasons, their \emph{algorithms} cannot really operate in real-time. In reality, a physical property changes continuously, but a robot's measurements can only be processed in certain time steps, which causes inaccuracy. 

All mentioned factors together might cause large uncertainty. Thus, the system needs to be able to accommodate or somehow compensate for it.

\subsection{Localization Problems}
Localization is not always the same. There are different localization problems depending on the environment and the use-case of the application. According to \citet{thrun:prob_robo} the following problems exist:

\paragraph{Local vs.\ Global Localization} If the (robot's) initial position is known when the localization starts, a so-called position tracking only is necessary, to know future positions. This is called \emph{local localization}.

In the case where the initial position is unknown, and the robot's position can by anywhere on the map, this is called \emph{global localization}. These types of algorithms are more difficult than position tracking. Global localization solves also the \emph{kidnapped robot} problem, which means that a robot is being placed on a different position during the position estimation. Of course, that is not very typically, but by solving this problem the robot is also able to recover if the algorithm is stuck in a state where it never will be able to approximate its true position.

\paragraph{Static vs.\ Dynamic Environments} In a \emph{static environment} the robot, i.e.\ the localizing object, is the only object that changes its position over time. All obstacles and other objects maintain their fixed position. In a \emph{dynamic environment} other objects besides the localizing object can change their position. Typical objects in a dynamic environment are humans, furnitures, like chairs or doors that sometimes are being moved, closed or opened.

\paragraph{Passive vs.\ Active Approaches} If the localization algorithm is able to control a robot's motion, this is called an \emph{active approach}. Through the ability to influence a robot's motion, the algorithm can induce movement of the robot if the localization is stuck and needs more information. For example from another point within a room, to decide what its position is. \emph{Passive} algorithms just observe the robot's motion, but cannot influence it. Usually, active approaches are built upon passive ones.

\paragraph{Single-Robot vs.\ Multi-Robot localization} \emph{Single-Robot} localization means, that the robot only needs to localize itself. \emph{Multi-Robot} localization requires of course multiple robots in the same area. It builds upon single-robot localization; thus, each robot can estimate its own position. New opportunities arise if the robots can recognize, each other and are able to communicate, to exchange their estimated positions.


\subsection{Performance Metrics}
To be able to compare different localization systems and algorithms, I introduce the following criteria, which are being recommended by \citet{IEEE:survey_wireless_indoor_pos}. Often the system's accuracy is the only metric being used, but also from my opinion as well, the system's accuracy is not sufficient enough for evaluation purposes.
\begin{itemize}
	\item \textbf{Accuracy} is the error of the estimated position. It is the Euclidean distance between the estimated and the real position.
	\item \textbf{Precision} takes the consistency into account. \citet{IEEE:survey_wireless_indoor_pos} defines it as the result of ``the cumulative probability function[s] (CDF) of the distance error'', for instance 90~\% location precision within 2~meters ($CDF(2m) = 0.9$).
	\item \textbf{Complexity} depends on several factors, such as the required hardware, the software complexity, the algorithms complexity, the necessary infrastructure, etc. However, \citet{IEEE:survey_wireless_indoor_pos} focus on software, i.e.\ computing complexity only.
	\item \textbf{Robustness} expresses the ability of a system to deal with wrong sensor data, unexpected values, or even no sensor data within a certain time interval.
	\item \textbf{Scalability} describes the possibility of a system to be extended or reduced in space, or a change of the density of the transmitters or measurement units.
	\item \textbf{Cost} depends on several factors. The financial costs are a very important factor. This depends on the hardware, implementation and installation costs. The energy consumption, which in the end also reflects in financial cost, is also an important factor. The factor time is also not negligible. The amount of time it takes to deploy and maintain such a system can be enormous.
\end{itemize}


\section{Localization Algorithms and Related Work}
In this section we focus on different localization algorithms. Some algorithms are discussed in more detail than others, depending on their closeness to our solution. After introducing the algorithm itself, we give an overview of related work, based on that algorithm. 

\subsection{Triangulation}\label{sec:fund_trilateration}
Triangulation is a well-studied localization method which relies on triangle's geometric properties \citep{IEEE:survey_wireless_indoor_pos, wang:bt_pos}. Triangulation is the hypernym of the two characteristics \emph{lateration} and \emph{angulation}.

\begin{figure}[width=0.9\textwidth, height=0.4\textheight]
	\subfloat[Lateration]{
  \includegraphics[width=0.45\textwidth]{figures/lateration}
  \label{fig:lateration}
}
\subfloat[Angulation]{
  \includegraphics[width=0.45\textwidth]{figures/angulation}
  \label{fig:angulation}
}

	\caption {Triangulation methods. Source:~\citep{wang:bt_pos}}
	\label{fig:triangulation}
\end{figure}

\subsubsection*{Lateration}
Lateration estimates the location based on the distances from the measuring unit to the transmitters. To precisely estimate a position, three base stations are required. The measuring unit measures the distance to each base station and puts a circle with the radius of the measured distance around the base station. The circles's common intersection point, is the exact position of the mobile unit. In 2D space at least three signals must be received, in 3D, four signals are required. According to \citet{wang:bt_pos}, one common intersection point exists only in theory. In practice, errors, due to obstacles and imperfect propagation models, influence the distances. Thus, no common intersection point exists. Figure~\ref{fig:lateration} depicts this case. Nevertheless, it is possible to estimate the mobile units position by using methods such as least-square, three-boarder, or the centroid-method \citep{wang:bt_pos, IEEE:survey_wireless_indoor_pos}.  

The distance for lateration can be estimated by different methods. According to \citet{IEEE:survey_wireless_indoor_pos}, the distance between the measuring unit and the transmitter is directly proportional to the signal's propagation time.
\emph{\ac{TOA}} uses this fact to estimate the distance by measuring the signal's travel time. Consequently, the transmitted signal contains the timestamp of its transmission. Thus, the clocks of all transmitters and measuring units, that are part of the system, need to be precisely synchronized, in order to perform a correct estimation, which is not a trivial problem. According to \citet{kotanen:exp_local_pos_bt}, a timing difference of  $1.0~\mu s$ causes an error of $\approx 300~\text{meters}$ in distance.

To overcome this disadvantage, a similar method called \emph{\ac{RTOF}} can be used. This method requires all components to be able to receive and transmit signals. The transmitter sends a signal, it is being received and immediately being returned, by the mobile unit. Thus, the transmitter sets the timestamp and compares it with its current timestamp, when the signal arrives. A relative time synchronization is sufficient enough. But \acs{RTOF} has another problem. The signal is being delayed by the processing time, which the unit requires to receive and to send it back to the original source.

The \emph{\ac{TDOA}} method calculates the distance from the different arrival times at the measuring units. Here, the measuring units need to be connected with each other to exchange the timestamps of the signals' arrival and to precisely synchronize their timestamps. The clocks' differences should not exceed tens of nanoseconds \citep{kotanen:exp_local_pos_bt}. The mobile unit, i.e.\ the transmitter, does not need to be synchronized. 

All of the above mentioned methods are based on time, and thus need precisely synchronized or relatively synchronized clocks. According to \citet{IEEE:survey_wireless_indoor_pos}, \acs{TOA} and \acs{TDOA} require a \acs{LOS} channel between transmitter and measuring unit in an indoor environment, because time and angle of the signals' arrival is being affected by the multi-path effect, which is a problem of radio propagation. Wireless technologies, such as \acl{BT} and WiFi do have a \emph{\ac{RSSI}} property, which expresses indirectly the signal's \ac{RX} power level, measured by the device \citep{kotanen:exp_local_pos_bt}. As a consequence, instead of measuring time to calculate the distance between sender and receiver, the loss of the signal strength on its path between the two units can be measured. In order to do that the measuring unit needs to know the initial \ac{RSS} of the emitted signal. As mentioned by \citet{IEEE:survey_wireless_indoor_pos}, different theoretical and empirical models can be used to translate the signals attenuation to a distance estimation, but these models also are not always reliable, due to multi-path fading and shadowing in indoor environment.

\subsubsection*{Angulation}
Instead of using distances for the positioning, angulation uses the angles to multiple measurement units, as shown in figure~\ref{fig:angulation}. In 2D space two angles, in 3D space three angles are sufficient for the positioning. But, to measure the angle of an arriving signal, special, directional antennas or arrays of antennas, are required. In the \emph{\ac{AOA}} method, the intersection point of the straights with the measured angles is the mobile units position. Here, highly precise angles need to be measured, which is a big problem in wireless environments due to multi-path reflections and shadowing, as already mentioned before \citep{IEEE:survey_wireless_indoor_pos, wang:bt_pos}.

\subsubsection*{Related Work}
\citet{wang:bt_pos} used Bluetooth to implement an indoor positioning system for mobile devices, like smartphones. The system is based on triangulation methods, i.e.\ on lateration, using \acs{RSS} for the distance estimation. During their research, they evaluated the least-square, three-border, and centroid method for the position estimation. They write, without mentioning any details, that all three methods deliver satisfying results, if the mobile unit receives accurate \acs{RSSI} readings and a proper path loss model is being used to calculate the distances. They also analyzed the effect of placing a human body between transmitter and measuring unit, and found out, that the signal is weakened by 6 -- 8~dB, which results in a few meters, due to the fact, that dB is a logarithmic unit. Due to the \ac{RSSI} values' fluctuation, they propose to filter the \acs{RSSI} readings, by an average or weighted filter, to improve the estimation accuracy.

\citet{oksar:bluetooth} also used Bluetooth, based on \acs{RSSI} readings, for their indoor localization. In contrast to \citet{wang:bt_pos}, the mobile units are the transmitters, and the measuring units are the fixed base stations with known position. Thus, the system topology is remote positioning. They defined a root-mean-square-error function, which compares ``the ratio of the square of distances to the base stations'' with ``the ratio of the signal levels'' \citep{oksar:bluetooth}. The used distances are being calculated for assumed discrete positions of the transmitter. As shown by \citet{oksar:bluetooth}, the better the assumed position matches the real position, the smaller is the result of the error function. They assumed, that the received \acs{RSS} decreases proportional to the squared distance. Thus, compared to \citet{wang:bt_pos}, they are not directly calculating the distance between the transmitter and the measuring unit. In their experiment they came up in the best case with a root-mean-square-error of $2.309~\text{meters}$ of localization accuracy. But they also mention, that in a real life application the points are most probably non-discrete points; thus, some changes are necessary.

\citet{hoflinger:acoustic} developed an acoustic indoor localization system called ASSIST, which stands for \emph{Acoustic Self-calibrating System for Indoor Smartphone Tracking}. According to \citet{hoflinger:acoustic}, the system can track a smartphone with an error of 25~cm. The smartphone sends out a high frequency acoustic signal using its speaker, which is not being recognizable for humans. The signal's ``amplitude $A$ of sound decreases with distance $r$ according to $A \sim 1/r$''. To detect the signals, fixed measuring units with a sensitive \ac{MEMS} microphone are being used. Despite the decreasing amplitude, they can detect the signals in 70\% up to 10~m distance. For the position estimation, the system uses the \acs{TDOA} method. The measuring units are connected via a wired network, which is at the same time the units' power supply. The network is used to synchronize the units' timestamps, and to exchange the measured timestamps. According to \citet{hoflinger:acoustic}, also the acoustic signals suffer from echoes and reflections, caused by multi-path propagation. To calculate a robust location, outliers are compensated by using \acs{PF} and \acs{KF}. They also mention, that the system can additionally use inertial smartphone sensors, such as accelerometer, gyroscope, and magnetometer, besides the acoustic infrastructure \citep{hoflinger:acoustic, hoflinger:assist}.

Further research projects based on triangulation methods are referenced by \citet{IEEE:survey_wireless_indoor_pos}.

\subsection{Scene Analysis}
Scene analysis localization algorithms, such as \emph{k-Nearest-Neighbors}, \emph{Neural Networks}, and \emph{Support Vector Machine}, are based on machine learning.

According to \citet{IEEE:survey_wireless_indoor_pos}, the use of such algorithms requires two stages, the offline and online stage. During the offline stage, fingerprints are collected at many positions in the area where the later localization should take place. Most often the fingerprints are \acs{RSS} based, which means, that at a specific position, the \acs{RSS} values of all received transmitters are collected. Then the fingerprints are stored in a database together with their corresponding location. The online phase is the actual run-time phase, where the localization takes place. The application also creates a fingerprint of the current received signals. The algorithm then compares the fingerprint with the fingerprints stored in the database during the offline stage, and returns the location of the best matching one.

As mentioned by \citet{IEEE:survey_wireless_indoor_pos}, the technique's main challenge is to deal with the signals that are being affected by their environment. The scene analysis's offline stage requires a lot of effort, especially in large environments, which is its major disadvantage. Another disadvantage is, that if the environment changes, the \acs{RSS} fingerprints may change, and thus, the offline stage needs to be repeated.

Our work follows different approaches; thus, we are not going into more detail on scene analysis. \citet{IEEE:survey_wireless_indoor_pos} gives a more detailed overview of the above mentioned algorithms.


\subsection{Proximity Method}
The proximity method, which is also known as cell-id tracking, only provides a relative location information. For the localization, the mobile device listens for signals from different transmitters. The transmitters' locations are known and can be identified by their identifier. As a result, the mobile device's location is \emph{next to the transmitter's location} with the strongest signal \citep{IEEE:survey_wireless_indoor_pos, wang:bt_pos}.

As mentioned by \citet{IEEE:survey_wireless_indoor_pos}, the algorithm used by the proximity method is very simple to implement. Additionally, it can use different wireless technologies at the same time, such as Bluetooth, WiFi, etc. The device just needs to \emph{listen} for all of them simultaneously. 

To get a precise position information, many transmitters need to be distributed within the environment. If there are too little, and their range is up to several meters, the location information is very rough \citep{kotanen:exp_local_pos_bt}. As mentioned before, wireless signal's are heavily influenced by the environment. Consequently, it may also happen, that the device's location is not optimal, due to the fact that the signals are not attenuated in the same strength, and thus, the device location is being associated with a transmitter that is further away, than another one.

\subsection{Kalman Filter}\label{sec:fund_kf}
The \emph{\ac{KF}}, and the \emph{\ac{EKF}}, are techniques for filtering and prediction of linear, and non-linear, gaussian systems with continuous space, such as an indoor robot localization system, based on the robot's motion and its measurements. As mentioned before, localization systems should be able to accommodate for uncertainty. Gaussian filters express the uncertainty of a state, e.g. a robots position, as the belief in the state. Beliefs are represented as multivariate normal distributions with mean $\mu$ and covariance $\Sigma$. % TODO Belief?

The algorithm is based on three probabilities. The \emph{state transition probability} $p(x_t | u_t , x_{t-1})$ is the probability for a state $x$ at time $t$ after the transition from state $x$ at time $t-1$ by applying the control $u$ at time $t$. Transferred to the above mentioned robot localization this means, that it expresses the belief in the robot's new position $x_t$ after a certain motion $u_t$ with being at position $x_{t-1}$. The \emph{measurement probability} $p(z_t|x_t)$ defines the belief in a certain measurement $z$ in state $x$ at time $t$. Thus, it expresses the belief in a measurement, e.g. a distance measurement $z_t$ to an obstacle, at position $x_t$. The last probability is the \emph{initial belief} $bel(x_0)$. Transferred to the example, it is the robot's initial position at time $t = 0$.

\begin{lstlisting}[
  float,
  floatplacement=H,
  mathescape,
  captionpos=b,
  caption=Algorithm for \acl{KF} based on \citet{thrun:prob_robo},
  label=lst:kf]
KalmanFilter($x_{t-1}$, $\Sigma_{t-1}$, $u_t$, $z_t$) {

  \\ Prediction
  $\hat{x}_t = A x_{t-1} + B u_t$
  $\hat{\Sigma}_t = A \Sigma_{t-1} A^T + B \Sigma_u B^T$

  \\ Correction
  $\hat{z}_t = C \hat{x}_t$
  $K = \hat{\Sigma}_t C^T (C\hat{\Sigma}_tC^T + \Sigma_z)^{-1}$
  $x_t = \hat{x}_t + K(z_t - \hat{z}_t)$
  $\Sigma_t = (I - KC)\hat{\Sigma}_t$

  return ($x_t$, $\Sigma_t$)
}
\end{lstlisting}
 % label: lst:kf

To provide a better overview, we now present the \acs{KF}'s linear implementation, with focus on the main parts, by not going into much detail. A much more detailed description can be found in \citet{thrun:prob_robo}. 
The \acs{KF}, shown in listing~\ref{lst:kf}, takes as first argument the current state $x_t$, e.g.\ representing the robot's position $x_t = (x, y)^T$. Together with the second argument $\Sigma_t$ the states uncertainty is being expressed, as multivariate gaussian $NDF(x_t, \Sigma_t)$. $u_t$ is the control vector, which represents the robots motion. The last argument $z_{t+1}$ is the current measurement; for instance, the measured distance to a wall.

The algorithm first performs the \emph{prediction} step (line 4, 5). The new state $\hat{x}_t$ and $\hat{\Sigma}_t$ is being predicted, based on its predecessor and the current control $u_t$, by taking the \emph{system model} $A$ and the \emph{motion model} $B$ into account.

Next, the \emph{correction} step is performed (line 8--11). The measurement $\hat{z}_t$ is predicted based on the predicted state $\hat{x}_t$. Then, the Kalman-Gain $K$ is calculated, by taking the \emph{measurement model} $C$, its covariance $\Sigma_z$, and the predicted covariance into account. It can be understood as weight, that is used in line 10,~11 to define how strong the predicted  state and the corresponding covariance is being influenced by the actual measurement.

In the end, the new state $x_t$ and its covariance $\Sigma_t$ are returned. The algorithm is iteratively called, during the localization process.

\subsubsection*{Related Work}
\acs{KF} can be used for indoor localization with bluetooth as demonstrated by \citet{kotanen:exp_local_pos_bt}. They use Bluetooth's \acs{RSS} to estimate the distance between the mobile unit and the transmitters. Then, they use \acl{EKF} to merge the mobile unit's measurements with the current state. According to \cite{kotanen:exp_local_pos_bt}, with knowing the uncertainties of the position and the measurements, \acs{KF} optimally minimizes the estimation error's variance. Their solution has an absolute positioning error of 3.76 meters.

\acs{EKF}'s state represents the mobile unit's three-dimensional position $x_t = (x, y, z)^T$. Their solution assumes a constant state; thus, no prediction takes place. Consequently, $\hat{x}_t = x_{t-1}$ and $\hat{\Sigma}_t = \Sigma_{t-1}$. $z_t$ is the vector off the distances between the mobile unit to all transmitters. The measurement model $C$ is non-linear, which is the reason for using \acs{EKF} instead of \acs{KF}.

As mentioned by \citet{kotanen:exp_local_pos_bt}, the ``main source of errors is unreliability of \acs{RSSI}''. Their position estimation is based on filtered values instead of raw values. But in fact, it only relies on the wireless measurements and does not include additional sensor data or other information, as usual in the robotic's field.

\subsection{Particle Filter}\label{sec:fund_pf}
The \ac{PF} is a non-parametric filter. The before introduced \acs{KF} is a parametric filter, it has fixed parameters which describe its functional form, the form of a gaussian. By contrast, \acs{PF} does not have fixed parameters, it approximates the functional form by a finite number of values, i.e.\ particles. The approximation's quality depends on the particle set's size. Figure~\ref{fig:pf_approx} illustrates the approximation of a posteriori distribution by particles, aka.\ samples. It also shows, that particle filter is well-suited for multi-modal beliefs \citep{thrun:prob_robo}.

\begin{figure}
	\includegraphics[height=0.45\textwidth]{figures/pf_approx}
	\caption{Shows the approximation of the posteriori distribution by samples. Source: \citep{thrun:prob_robo}}
	\label{fig:pf_approx}
\end{figure} 

According to \citet{thrun:prob_robo}, the idea is the posterior's $bel(x_t)$ representation by a random set of samples, drawn from the posteriori. Each sample (i.e.\ particle) $x$ at time $t$ is a hypothetical state in state space, i.e.\ in true world. The particle set $\chi_t = \left\{ x^{[1]}_t, x^{[2]}_t, \ldots, x^{[m]}_t \right\}$ is the distribution's approximation, containing $M$ particles. The \acs{PF}'s algorithm, depicted in listing~\ref{lst:pf}, transforms the $bel(x_{t-1})$, represented by the particle set $\chi_{t-1}$, by integrating the least control $u_t$ and measurement $z_t$ into the current $bel(x_t)$. Therefor, the algorithm creates first an empty, temporary particle set $\bar{\chi}_t$. Then, the states from $\chi_{t-1}$ are being updated, by integrating the current control $u_t$. Formally, this is done by sampling the new state $x_t$ from the \emph{state transition probability} $p(x_t|u_t, x^{[m]}_{t-1})$ (line 5). Afterwards, an \emph{importance factor} $w^{[m]}_t$ for the new sample $x^{[m]}_t$ is being calculated (line 6). It is defined as $p(z_t|x^{[m]}_t)$, which is the probability for the measurement $z_t$ given the new state hypothesis, representing a value in true world. The new state hypothesis and its importance factor are stored together, as tuple, in the temporary set $\bar{\chi}_t$ (line 7). The weighted set approximates the $bel(x_t)$.

Then, the resampling, i.e.\ importance sampling, transforms the weighted set, based on their weights into a new set, which represents the new distribution. Therefor, $M$ particles are drawn according their weight and being inserted into the new set $\chi_t$ (line 14, 15). During this step some particles are duplicated, usually the ones with high weight, and others, the ones with low weight, are lost. \citet{thrun:prob_robo} compares it with the Darwinian idea of \emph{survival of the fittest}. According to them, ``it refocuses the particle set to regions in state space with high posteriori probability''.
 
Instead of the drawing, it is also possible to have a weighted set $\chi_t$. In every recursion, the weights of the existing particles are multiplied with their new weight. The method's disadvantage is, that a larger particle set is being required to reach the same approximation quality, because the particles do not move, they keep their position in state space for ever. For this method the particles need to be initialized with weight $1$.
 
\acs{PF}'s can be adaptive, for instance based on the available processing power; thus, the particle set is then being increased or decreased \citep{thrun:prob_robo}.

\begin{lstlisting}[
  float,
  floatplacement=H,
  mathescape,
  captionpos=b,
  caption=Algorithm for \acl{PF} based on \citet{thrun:prob_robo},
  label=lst:pf]
ParticleFilter($\chi_{t-1}$, $u_t$, $z_t$) {

  $\bar{\chi}_t = \emptyset$
  for $m = 1$ to $M$ {
    sample $x^{[m]}_t \sim p(x_t|u_t, x^{[m]}_{t-1})$
    $w^{[m]}_t = p(z_t|x^{[m]}_t)$
    add $\langle x^{[m]}_t, w^{[m]}_t \rangle$ to $\bar{\chi}_t$
  }


  \\ resampling
  $\chi_t = \emptyset$
  while size of $\chi_t$ less $M$ {
    draw $i$ with probability $\propto w^{[m]}_t$
    add $x^{[i]}_t$ to $\chi_t$
  }

  return $\chi_t$
}
\end{lstlisting}
 % label= lst:pf

\subsubsection*{Monte Carlo Localization}\label{sec:fund_mcl}
\ac{MCL} is a popular localization algorithm often used in the indoor mobile robotics field. The algorithm can deal with a broad range of the before discussed localization problems, especially with the local and global localization problem.

The basic algorithm, shown in listing~\ref{lst:mcl}, is very similar to the \acs{PF}, presented in listing~\ref{lst:pf}. Line~5 is being substituted by a method called \texttt{sample\_motion\_model}, which samples the new hypothesis by integrating the control $u_t$ with respect to the robot's motion and its uncertainties during the execution. Furthermore, line~6 is being substituted by the \texttt{measurement\_model} function, which depends on the used sensor. It uses the sensor measurements and their uncertainty to calculate the particle's importance factor. It additionally takes the environment's map into account. Usually, the map is being used to calculate the importance factor based on the measurement $z_t$; for instance, the distance to a landmark, where the landmark's position is stored in the map. But a map can also be used to detect particles at impossible positions, and thus, to reduce their importance factor. A very detailed description about \acl{MCL} and variations of the algorithm is provided by \citet{thrun:prob_robo}.

A visual example of, how a robot's global localization using \acs{MCL} works, is shown in figure~\ref{fig:mcl}. The three images show the particles, i.e.\ the posteriori $bel(x_k)$, over time. At the beginning uniform random particles are being spread over the whole state space, because the robot actually does not know where it is, and thus, could be positioned anywhere. After moving and gathering sensor data, the particles concentrate in certain regions with higher belief for the robot's position. Thus, the posteriori has changed. In the end, the particles have concentrated on one spot; hence, the robot's confidence of being at this position is very high \citep{thrun:prob_robo}.

\begin{lstlisting}[
  float,
  floatplacement=H,
  mathescape,
  captionpos=b,
  caption=Basic \acl{MCL} algorithm based on \citet{thrun:prob_robo},
  label=lst:mcl]
mcl($\chi_{t-1}$, $u_t$, $z_t$, map) {

  $\bar{\chi}_t = \emptyset$
  for $m = 1$ to $M$ {
    $x^{[m]}_t$ = sample_motion_model($u_t$, $x^{[m]}_{t-1}$)
    $w^{[m]}_t$ = measurement_model($z_t$,$x^{[m]}_t$, map)
    add $\langle x^{[m]}_t, w^{[m]}_t \rangle$ to $\bar{\chi}_t$
  }


  \\ resampling
  $\chi_t = \emptyset$
  while size of $\chi_t$ less $M$ {
    draw $i$ with probability $\propto w^{[m]}_t$
    add $x^{[i]}_t$ to $\chi_t$
  }

  return $\chi_t$
}
\end{lstlisting}



\begin{figure}[width=0.9\textwidth, height=0.4\textheight]
	\subfloat[]{
  \includegraphics[width=0.3\textwidth]{figures/mcl_1}
}
\subfloat[]{
  \includegraphics[width=0.3\textwidth]{figures/mcl_2}
}
\subfloat[]{
  \includegraphics[width=0.3\textwidth]{figures/mcl_3}
}

	\caption{Visual example of the \acs{MCL} algorithm. Source:~\citep{thrun:prob_robo}}
	\label{fig:mcl}
\end{figure}


\subsubsection*{Related Work}

\citet{Siddiqui:tracking} propose a localization system that uses \acs{PF} and \acs{RSS}-based trilateration of WiFi signals, to track a mobile unit. The approach just relies on sensing \acs{RSS} values, no other sensors are being used. First, they collect the \acs{RSS} values in their environment by war-walking. They are stored together with their location in a lookup table. For their MATLAB simulation they generated a random path with 100~points using the before collected values. The path is tracked during the simulation, just based on the \acs{RSS} values. Their simulation simulates a notebook equipped with WiFi, which is tracked. Thus, their \acs{PF}'s implementation does not sample a new state by integrating the control $u_t$ as shown in listing~\ref{lst:pf} (line~5). They used two different weighting functions during their experiment, firstly, a \emph{Gaussian weighting function}, based on the Euclidian distance, and secondly, a \emph{Triangular weighting function} (line 6). According to \citet{Siddiqui:tracking}, they achieved an accurate localization of $\approx$~3~m, as a result of their simulation. During the simulation they assumed, a static environment with sparse activity, and a smooth and continuous motion. They also assumed that no large variations in the signals' strength occur over short distances. During their signal recordings they removed the first and last 10\% of the measurements, when the person placed the measuring unit at the position and picked it up again, because they wanted their values to be independent of a human in proximity. Additionally, they neglected outlaws.

\citet{straub:pf} implemented an indoor pedestrian localization and tracking system based on \acs{PF}, which uses a system called \emph{PiNav}, developed at the institute where Straub wrote his thesis. It extracts the person's motion in form of step frequency, step size, and heading. The person needs to be equipped with the hip-mounted PiNav-System, which includes an accelerometer, gyroscope and a compass. Additionally, his solution uses an accessibility map, which is based on a neural network, to learn which space on the map is accessible by pedestrians. By fusing the pedestrian's motion and the accessibility map in a \acs{PF}, his solution achieves an average accuracy of 1.1~m \citep{straub:pf}.

\citet{wang:wlan} proposes a \acs{RSS}-based WiFi positioning system using particle filter to fuse the location estimated via WiFi with additional map and accelerometer data. Their system uses the above mentioned scene analysis approach, especially the k-Nearest Neighbors algorithm for the positioning. During the offline stage they collect fingerprints of the WiFi signal in one meter distance to the access points. During the localization, i.e.\ the online stage, the algorithm compares the fingerprints of the collected WiFi signals with the stored ones from the offline stage and returns a location estimation. According to \citet{wang:wlan}, the 3-axis accelerometer's values can be used in theory, to estimate the traveled distance by integrating the values. But, due to the low signals and the sensors' noise it does not work in praxis. Thus, they use a zero-crossing approach to detect steps and estimate their size. The approach detects a step if the vertical acceleration crosses the zero-line. Additionally, their solution uses the environment's map to sort out impossible particles, like particles crossing a wall.
Their simulation results show, that by using map and accelerometer in addition to kNN, their location error improves by 40\%. Their solution is also 30\% better than the \ac{KF}'s result. During their real world tests, they achieved a mean error of 4.3~m with a standard deviation of 2.8~m \citep{wang:wlan}.

\citet{siddiqi:experiments_mcl_wifi} uses \acs{MCL} to globally localize a \emph{Pioneer 2DX} robot in an indoor environment using the robots odometers and the measured \acs{RSS} of WiFi signals. The algorithm's action model, i.e.\ motion model, is based on the data measured by the odometers. They deliver data for short distances with a very good accuracy. The motion is split up into translations and rotations. Their action model samples the motion with different uncertainties for translation and rotation. The observation model, i.e.\ measurement model, uses a signal strength map, which is split up into a grid with squared fields with a size of 0.5~m. Before localization can take place, for each grid cell the \acs{RSS} of each WiFi access point needs to be measured, to store \acs{RSS} mean and standard deviation. Due to the fact, that \citet{siddiqi:experiments_mcl_wifi} assume that the signal attenuation does not change on short distances, they just measured it for a few fields, and assumed the same value for the neighbor fields. During the localization, the actual \acs{RSS} values are compared with the collected one for the importance factor. Additionally, their observation model weeds out particles, by using the map constraints, but without taking the robot's orientation into account. According to \citet{siddiqi:experiments_mcl_wifi}, by using just the map constraints without the \acs{RSS} comparison, very good localization results can already be achieved.
During their tests they found out, that by increasing the amount of access points, the location error decreases \citep{siddiqi:experiments_mcl_wifi}. They mention, that their approach should also work for localization of humans instead of robots. But they also note, that a robot's motion model is simpler than the one of a human. According to \citet{siddiqi:experiments_mcl_wifi}, they achieved during most test cases an accurate localization with an error of $\approx 2~m$.
