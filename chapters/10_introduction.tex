\chapter{Introduction} \label{chap:intro}

Would it not be great if at the airport we could be guided from the arrival to the departure gate? Would it not be fantastic, if at a huge conference or trade fair your smartphone guides you from one interesting stand to the next one? Would it not be awesome, to compose a shopping list in your favorite grocery store's app, and then being guided on the shortest and fastest path through the store?

Location-based applications are by many people taken for granted and seen as natural aid in their life. About two decades ago, the probably most popular location-based applications, \ac{GPS} based navigation systems, drastically improved our possibilities, by providing an alternative to the old paper map. Today, \ac{GPS} based applications are an essential feature of smartphones. We use them on a daily basis to determine our position in new cities and on trails, or to track our sports activities. Importantly, all of them only work outdoors, because \ac{GPS} requires \ac{LOS}, from the smartphone to the satellites for accurate location estimation.

In the area of mobile indoor robotics, indoor self-localization is a well known problem. Robots are equipped with sensors, such as odometers, laser scanners, or ultrasonic sensors, to determine their position. Humans are typically not equipped with sensors, but smartphones, our permanent companions, contain lots of them, to perceive our environment.

Actually, approaches for indoor self-localization with smartphones, based on WiFi or Bluetooth, do exist, but until now, they did not manage to get established in our, respectively the everyday life. From my point of view it is a classical chicken-and-egg problem. Most people do not know that technically indoor localization is possible, and thus the owners of buildings do not see a reason to provide the necessary infrastructure. An additional confounding factor is that the needed infrastructure is relatively expensive. Consequently, once the infrastructure has become cheaper and the technical feasibility is more widely known, applications of this technology will emerge.

Apple contributed to the solution of both problems by introducing a new technology called \emph{iBeacon}. This was presented at their yearly \ac{WWDC} in June~2013 \citep{apple:wwdc_2013_bruins}. iBeacons are small hardware chips equipped with a \ac{BLE} module. Adequately equipped smartphones can pick up the \ac{BLE} signals and estimate the proximity to a beacon. \citet{apple:ibeacon_site} promotes the technology on their iBeacon developer website with the following use-cases: ``From welcoming people as they arrive at a sporting event to providing information about a nearby museum exhibit, iBeacon opens a new world of possibilities for location awareness, and countless opportunities for interactivity between iOS devices and iBeacon hardware.'' Finally, the press also added the above mentioned use-case \emph{indoor navigation}, such as navigation from arrival hall to the departure gate at an airport, or the navigation from one interesting booth to the next at a conference or trade fair.


\section{Requirements}
From my point of view an indoor self-localization system should fulfill the following requirements. The system should have the possibility to be used with any of the most common devices. Therefore, it needs to be cross-platform, i.e.\ across operating systems. Additionally, the system should be of low cost for both, the relevant building or business owner as well as the user. Consequently, the system’s infrastructure should be as simple as possible. The system should also be easy-to-deploy and cause little maintenance costs. Besides that, the users’ device and the application's algorithm should be as independent as possible from the remaining infrastructure. Thus, localization should take place on the user’s device.


\section{Idea} % TODO MB? The Idea for this thesis
The idea was to design, implement and evaluate an indoor self-localization system with smartphones using iBeacons, and which fulfills the above mentioned requirements.

To that purpose, commonly available smartphones equipped with a \acs{BLE} module were used. The solution is based on \ac{MCL}. The smartphones' built-in sensors were used to measure the user's motion. The algorithm requires also a map of its environment. Additionally, the proximity to known beacons is integrated into the position estimation.

The designed solution can run on every platform. iBeacons are relatively cheap and require no additional infrastructure, such as a power supply or network connection. They can be powered with a small battery or a coin cell over months. The deployment is very simple, as well. The beacons just need to be distributed over the area, where the localization should take place. Additionally, an initial calibration step needs to be performed. If the beacons' environment does not change, no maintenance is necessary. The user's device is completely independent. The whole position estimation takes place on the user's device. No connection between smartphone and the remaining infrastructure is being established.


\section{Structure}
The document is divided into seven chapters. First, an overview about localization in general is given. Chapter~\ref{chap:fundamentals} covers localization algorithms and related work. Then, Chapter~\ref{chap:ibeacons} gives a detailed insight into iBeacon technology, including \acl{BLE}, configuration parameters, the \ac{API} provided by Apples iOS 8.1 operating system, and the signal's quality. Afterwards, a close look on the smartphones' built-in sensors, the provided \acsp{API} and the provided data is taken in Chapter~\ref{chap:sensors}, including an evaluation at the chapter's end. After providing the basis, the solution's implementation is presented in Chapter~\ref{chap:pf}. This proof of concept is evaluated in Chapter~\ref{chap:evaluation}. Finally, research is concluded in Chapter~\ref{chap:conclusion}, including ideas, how the project could be developed further.
