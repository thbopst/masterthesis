\chapter{Introduction} \label{chap:intro}

Wouldn't it be great, to being guided on the airport while hurrying from the arrival to the departure gate? Wouldn't it be fantastic, to have the smartphone guiding oneself on a huge trade fair from one interesting fair stand to the next one? Wouldn't it be awesome, to enter a shopping list in oneself's favorite grocery store's app, and then being guided through the store, on the shortest and quickest path?

%\section{Problem}
Location-based applications, have a permanent position in our life. A few years ago, the probably most popular location-based application, \acs{GPS} based navigation systems, drastically improved our navigation skills on street, by replacing the old paper maps. Today, \acs{GPS} based applications are an elementary feature of smartphones. We use them on a daily basis, to determine our position in new cities and on trails, or to track our sports activities. But, all of them works just outdoors, because \acs{GPS} requires \acl{LOS}, for accurate location estimation.

In the area of mobile indoor robotics, indoor self-localization is a well known problem. Robots are equipped with sensors, such as odometers, laser scanners or ultrasonic sensors, to determine their position. Humans are typically not equipped with sensors, but smartphones, our permanent companions, contain lots of them, to perceive our environment.

Actually, approaches for indoor self-localization with smartphones based on WiFi or Bluetooth do exist, but until now, they did not manage to get established in our, resp.\ the mainstream's, daily life. From my point of view it is a chicken-and-egg problem. Most people do not know that technically indoor localization is possible and thus, the owners of buildings feel no pressure to provide the necessary infrastructure. On the other side building owner's will not invest on their part, due to the expensive infrastructure. Consequently, people should get to be aware of these possibility, or the infrastructure needs to get cheaper.

Apple addressed both problems at the same time by introducing a new technology, called \emph{iBeacon}, at their yearly \ac{WWDC} in June~2013 \citep{apple:wwdc_2013_bruins}. iBeacons are small hardware chips equipped with a \ac{BLE} module. Capable smartphones can receive their \acs{BLE} signals to estimate the proximity to a beacon. \citet{apple:ibeacon_site} promotes the technology, on their iBeacon developer website, with the following use-cases: ``From welcoming people as they arrive at a sporting event to providing information about a nearby museum exhibit, iBeacon opens a new world of possibilities for location awareness, and countless opportunities for interactivity between iOS devices and iBeacon hardware.'' Finally, the press made them popular by adding the use-case \emph{indoor navigation}, to solve problems as the above stated ones.

% zu kompliziert, läufte fordern sie nicht genug, jedoch auch zu teuer und komplex für shops und anbieter
% additionally use sensors in smartphone
% location-based applications broad appeal (navigation, robotics, gaming, asset tracking, ...)
% no GPS, loss of signal, or very bad accuracy
\section{Requirements}
From our point of view an indoor self-localization system should satisfy the following requirements. The system should be usable with common devices. Therefore, it needs to be cross-platform. Additionally, the system should be of low cost for both, the building owner and the user. Consequently, the system's infrastructure should be as simple as possible. The system should also be easy-to-deploy, and cause less maintenance costs. Besides that, the user's device and application's algorithm should be as independent as possible from the remaining infrastructure. Thus, localization should take place on the user's device.

\section{Idea}
The idea is to design, implement and evaluate an indoor self-localization systems with Smartphones using iBeacons, which satisfies the before mentioned requirements.

Therefore, common smartphones equipped with a \acs{BLE} module can be used. Our solution is based on \ac{MCL}. The smartphone's build-in sensors are being used to measure the user's motion. The algorithm requires also a map of its environment. Additionally, the proximity to known beacons is being integrated into the position estimation.

The designed solution can run on every platform. iBeacons are relatively cheap and require no additional infrastructure, such as a power supply or network connection. They can be powered with a small battery or a coin cell over months. Also, the deployment is very simple. The beacons just need to be spread over the area, where the localization should take place. Additionally an initial calibration step needs to be performed. If the beacons environment does not change, no maintenance is necessary. The user's device is completely independent. The whole position estimation takes place on the user's device. No connection between smartphone and the remaining infrastructure is being established.


%Use approach from robotics.
%
%Ein Grund Q/A [Apple intro], says it is not possible to show precise location.
%
%UseCase:
%\begin{itemize}
%  \item (assets management, staff tracking,) indoor tourist guiding in museums, train stations, airport, shopping complex \citep{wang:bt_pos}, mobile guides, indoor navigation
%  \item example ADAMANT (airport travelers guide) \citep{wang:bt_pos}
%  \item In robotic indoor localization alter hut
%\end{itemize}
%
%Requirements:
%\begin{itemize}
%\item low cost
%\item public available device
%\item cross platform
%\item every thing on device \citep{wang:bt_pos}
%\item easy-to-deploy \citep{wang:bt_pos}
%\end{itemize}
%
%Map with beacon positions

\section{Structure}
The document is split up into seven chapters. First, we give an overview about localization in general, localization algorithms, and related work, in chapter~\ref{chap:fundamentals}. Then, chapter~\ref{chap:ibeacons} gives a detailed insight into iBeacon technology, including \acs{BLE}, configuration parameters, the \acs{API} provided by Apples iOS 8 operating system, and the signal's quality. Afterwards, we are taking a close look, in chapter~\ref{chap:sensors}, on the smartphones build-in sensors, the provided \acsp{API} and the provided data, which we are going to evaluate at the chapter's end. After founding the basis, we present our algorithms implementation in chapter~\ref{chap:pf}, which is a proof of concept, which is being evaluated in chapter~\ref{chap:evaluation}. Finally, we conclude our research in chapter~\ref{chap:conclusion}, including ideas, how the project could be developed further.


