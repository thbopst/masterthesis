\chapter{Localization Algorithm} \label{chap:pf}
Our solution is based on \acl{MCL}, as presented in chapter~\ref{chap:fundamentals}. In this chapter we first outline the reasons for choosing \acs{MCL} instead of another algorithm. Next we describe the algorithms \emph{motion model}, which is on one side responsible for tracking the users motion by combining different sources, introduced in chapter~\ref{chap:sensors}, and on the other side for the sampling from the motion model. Afterwards, we give a detailed insight in our solution's \emph{measurement model}. It tracks the beacon's signals, resp.\ the distances to the beacons, and implements the \acs{PF}'s importance factor calculation. In the End, we explain the algorithm's implementation, which builds upon the motion and measurement model.
\section{Design Decision}

% small pre-deployment effort (scene analysis)
% integration of map (hui wang)
% easy to implement (\thrun)
% multi-modal distribution (whereas kalman just can normal distribution \cite{siddiqi})
% keine zusätzliche Infrastruktur, keinen zusätzlichen Server
% sensor fusion 
% globale localization -> Kidnapping
% Filter soll funktionieren wenn keine Meßwerte vorliegen bzw. auch wenn nur bewegung da ist
% geeignet für localisierung mit landmarken




% Decision PF (Map)
% Particle Filter

% small pre deployment effort
% no effort for RF maps such as scene analaysis

% according to hui wang, KF only for linear, and EKF must be more or less linear, its impossible to integrate map information

% ease of implementation \citep{siddiqui}, fast-realtime properties
% mcl easy implement \thrun
% multimodal whereas kalman just can normal distribution \cite{siddiqi}


\begin{figure}
\def\svgwidth{0.9\textwidth}
\input{algo_architecture.pdf_tex}
%%\inputSVG[\def\svgheight{0.45\textwidth} \small]{architecture}
%
\caption{Architecture}	
\end{figure}



\section{Motion Model}

\section{Measurement Model}

\section{\acl{MCL}}
