\chapter{Conclusion} \label{chap:conclusion}
At the beginning three example use-cases for indoor self-localization, which could provide benefit to our lives, were mentioned. These and other use-cases are now, from my perspective, achievable goals, because as of now, the problem was rather the localization approach than the large beacon deployments, as recently shown at the \emph{South By South West (SXSW)} music and film festival in Austin (TX, USA). There the festival organizers deployed more than 1000\,beacons to enhance the festival with social networking features \citep{mashable}.

In this work I presented an indoor self-localization approach for smartphones using iBeacons to exactly solve this problem. The solution is a cross-platform solution, that works with common smartphones, equipped with built-in sensors and a Bluetooth Smart, i.e.\ \acl{BLE}, chip.

The system's complexity is very low. Besides the smartphones and the beacons, no additional hardware or infrastructure is required, which also reduces the initial and maintenance costs. The whole localization takes place on the user's smartphone. Furthermore, the location estimation is very robust compared to other approaches.

Because of high uncertainty of the \acs{RSS} based distance estimation to the deployed known beacons, the solution relies not only on the wireless signals. In addition, it uses the smartphone's built-in sensors to determine the user's motion. Furthermore, the building's map, which represents its free and occupied space, is integrated. Finally, the solution combines all three sources by using a \acl{PF} to estimate the user's location and to reduce the overall uncertainty.

As a result, the approach solves the global localization problem, and is also able to detect and to recover from wrong position estimations. During the experiments, a user's stationary location with a mean error of 2.29\,m, was achieved. Furthermore, it was possible to track a user's path in the test environment.


\section{Future Work} \label{sec:future}
There are still a few open questions. Some of them are merely interesting, others are necessary for future improvement.

Companies such as Walmart and Walgreens announced in May\,2014, that they will deploy beacons, by using General Electric's in-store LED lighting fixtures, which additionally act as beacons. Thus, it would be interesting to know, if the localization could be improved by deploying beacons at the building's ceiling, i.e.\ above the users \citep{9to5mac}.

Additionally, it would be interesting to know, if a better location estimation could be achieved by using Apple's new iPhone\,6/ iPhone\,6\,Plus, which is shipped with a new motion coprocessor.

The new iPhone also is equipped with a new sensor, a barometer. Thus, it would be worthwhile to extend the solution to three-dimensional space and to integrate \acl{CM}'s floor ascending and descending capability.

Furthermore, it would be good to know, if there are differences in the beacon signal's quality, especially if a more accurate proximity estimation would be possible, by using beacons from other manufacturers.

Another very interesting topic, maybe a research topic, would be to find out, if the solution would benefit from solving the multi-``robot'' problem, by turning the device itself into a beacon, and thus having the additional possibility of communicating with other devices.

But of course, the most interesting would be to see how good the application would perform in large deployments, for instance at a trade fair.